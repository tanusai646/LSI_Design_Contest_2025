%% 4. 「技術研究報告」
\documentclass[technicalreport]{ieicej}
%\usepackage[dvips]{graphicx}
%\usepackage[dvipdfmx]{graphicx,xcolor}
\usepackage[fleqn]{amsmath}
\usepackage{newtxtext}% 英数字フォントの設定を変更しないでください
\usepackage[varg]{newtxmath}% % 英数字フォントの設定を変更しないでください
\usepackage{latexsym}
%\usepackage{amssymb}

\jtitle{テスト}
\jsubtitle{テスト2}
\etitle{test}
\esubtitle{test2}
\authorlist{%
\authorentry[imamura.yuuki475@mail.kyutech.jp]{今村 優希}{Yuki Imamura}{kyutech}
\authorentry[kawasaki.taiga000@mail.kyutech.jp]{川崎 大雅}{Taiga Kawasaki}{kyutech}
% \authorentry[メールアドレス]{和文著者名}{英文著者名}{所属ラベル}
}
\affiliate[kyutech]{九州工業大学情報工学部 情報・通信工学科 3年\\ 
  福岡県飯塚市川津680-4}
  {Kyushu Institute of Technology\\ School of Computer Science and System Engineering\\ Department of Computer Science and Networks}
%\affiliate[所属ラベル]{和文勤務先\\ 連絡先住所}{英文勤務先\\ 英文連絡先住所}
\jalcdoi{???????????}% ← このままにしておいてください

\begin{document}
\begin{jabstract}
%和文あらまし
テスト
\end{jabstract}
\begin{jkeyword}
%和文キーワード
VAE, FPGA, 画像認識
\end{jkeyword}
\begin{eabstract}
%英文アブストラクト
test
\end{eabstract}
\begin{ekeyword}
%英文キーワード
VAE, FPGA
\end{ekeyword}
\maketitle

\section{はじめに}
近年,無線通信技術は飛躍的に向上しており,5G通信の普及が進んでいる.
5Gは従来の4Gなど通信規格と異なり,「高速大容量」「低遅延」「多数同時接続」の3つの特徴を備えており,その中でも「低遅延」と「多数同時接続」は新たな通信環境を構築する上で重要な軸となっている\cite{5g}.

従来の4G通信は,人が使用するスマートフォンや携帯に焦点を当てていた.
しかし,5Gでは車両,ドローン,センサなどのIoT機器が大量にネットワークに接続されることを前提としている.
このような環境においては,従来のクラウド中心の処理方式では,トラヒック増加による遅延や負荷集中が生じる可能性があり,5Gの利点を十分に発揮できない問題がある.
また,今現在のIT業界ではクラウドが主流で,処理の多くを一つ(もしくは複数の)コンピュータで行うという構造である.
多くの端末から取得したデータをクラウドのみで処理を行うのはある程度限界があり,またトラヒック量が増加して,5Gのメリットを享受できないという問題が発生すると考えられる.

このような課題を解決するため,エッジコンピューティングという技術が近年注目され始めている.
エッジコンピューティングとは,従来はクラウドで行っていた処理の一部を,ユーザ端末(スマートフォンやIoT機器)の近い位置である基地局やその至近に設置されているサーバなどでデータ処理を行う技術である\cite{edge-com}.
この技術を用いることでクラウドにかかる処理をエッジコンピューティングで分散することが可能で,通信のトラヒック量,5Gの特徴のひとつである「低遅延」に貢献することも可能である.

そこで,エッジコンピューティングの実現をVAEとFPGAを用いて実現することを考えた.
(本文を書きながら続きを記述予定)

\section{学習方法}
何かしら書く

\section{回路設計}

\section{システム評価}

\section{出力結果}

\section{終わりに}

%\bibliographystyle{sieicej}
%\bibliography{myrefs}
\begin{thebibliography}{99}% 文献数が10未満の時 {9}
\bibitem{5g} 
森川博之, 5G次世代移動通信規格の可能性, 岩波書店, 
\bibitem{edge-com} 
田中裕也, 高橋紀之, 河村龍太郎, "IoT時代を拓くエッジコンピューティングの研究開発", NTT技法ジャーナル, vol.27, no.8, pp.59-63, 2015.
\end{thebibliography}

\end{document}
